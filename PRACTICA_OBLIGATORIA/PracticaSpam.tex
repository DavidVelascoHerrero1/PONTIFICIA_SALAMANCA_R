% Options for packages loaded elsewhere
\PassOptionsToPackage{unicode}{hyperref}
\PassOptionsToPackage{hyphens}{url}
%
\documentclass[
]{article}
\usepackage{amsmath,amssymb}
\usepackage{lmodern}
\usepackage{iftex}
\ifPDFTeX
  \usepackage[T1]{fontenc}
  \usepackage[utf8]{inputenc}
  \usepackage{textcomp} % provide euro and other symbols
\else % if luatex or xetex
  \usepackage{unicode-math}
  \defaultfontfeatures{Scale=MatchLowercase}
  \defaultfontfeatures[\rmfamily]{Ligatures=TeX,Scale=1}
\fi
% Use upquote if available, for straight quotes in verbatim environments
\IfFileExists{upquote.sty}{\usepackage{upquote}}{}
\IfFileExists{microtype.sty}{% use microtype if available
  \usepackage[]{microtype}
  \UseMicrotypeSet[protrusion]{basicmath} % disable protrusion for tt fonts
}{}
\makeatletter
\@ifundefined{KOMAClassName}{% if non-KOMA class
  \IfFileExists{parskip.sty}{%
    \usepackage{parskip}
  }{% else
    \setlength{\parindent}{0pt}
    \setlength{\parskip}{6pt plus 2pt minus 1pt}}
}{% if KOMA class
  \KOMAoptions{parskip=half}}
\makeatother
\usepackage{xcolor}
\usepackage[margin=1in]{geometry}
\usepackage{color}
\usepackage{fancyvrb}
\newcommand{\VerbBar}{|}
\newcommand{\VERB}{\Verb[commandchars=\\\{\}]}
\DefineVerbatimEnvironment{Highlighting}{Verbatim}{commandchars=\\\{\}}
% Add ',fontsize=\small' for more characters per line
\usepackage{framed}
\definecolor{shadecolor}{RGB}{248,248,248}
\newenvironment{Shaded}{\begin{snugshade}}{\end{snugshade}}
\newcommand{\AlertTok}[1]{\textcolor[rgb]{0.94,0.16,0.16}{#1}}
\newcommand{\AnnotationTok}[1]{\textcolor[rgb]{0.56,0.35,0.01}{\textbf{\textit{#1}}}}
\newcommand{\AttributeTok}[1]{\textcolor[rgb]{0.77,0.63,0.00}{#1}}
\newcommand{\BaseNTok}[1]{\textcolor[rgb]{0.00,0.00,0.81}{#1}}
\newcommand{\BuiltInTok}[1]{#1}
\newcommand{\CharTok}[1]{\textcolor[rgb]{0.31,0.60,0.02}{#1}}
\newcommand{\CommentTok}[1]{\textcolor[rgb]{0.56,0.35,0.01}{\textit{#1}}}
\newcommand{\CommentVarTok}[1]{\textcolor[rgb]{0.56,0.35,0.01}{\textbf{\textit{#1}}}}
\newcommand{\ConstantTok}[1]{\textcolor[rgb]{0.00,0.00,0.00}{#1}}
\newcommand{\ControlFlowTok}[1]{\textcolor[rgb]{0.13,0.29,0.53}{\textbf{#1}}}
\newcommand{\DataTypeTok}[1]{\textcolor[rgb]{0.13,0.29,0.53}{#1}}
\newcommand{\DecValTok}[1]{\textcolor[rgb]{0.00,0.00,0.81}{#1}}
\newcommand{\DocumentationTok}[1]{\textcolor[rgb]{0.56,0.35,0.01}{\textbf{\textit{#1}}}}
\newcommand{\ErrorTok}[1]{\textcolor[rgb]{0.64,0.00,0.00}{\textbf{#1}}}
\newcommand{\ExtensionTok}[1]{#1}
\newcommand{\FloatTok}[1]{\textcolor[rgb]{0.00,0.00,0.81}{#1}}
\newcommand{\FunctionTok}[1]{\textcolor[rgb]{0.00,0.00,0.00}{#1}}
\newcommand{\ImportTok}[1]{#1}
\newcommand{\InformationTok}[1]{\textcolor[rgb]{0.56,0.35,0.01}{\textbf{\textit{#1}}}}
\newcommand{\KeywordTok}[1]{\textcolor[rgb]{0.13,0.29,0.53}{\textbf{#1}}}
\newcommand{\NormalTok}[1]{#1}
\newcommand{\OperatorTok}[1]{\textcolor[rgb]{0.81,0.36,0.00}{\textbf{#1}}}
\newcommand{\OtherTok}[1]{\textcolor[rgb]{0.56,0.35,0.01}{#1}}
\newcommand{\PreprocessorTok}[1]{\textcolor[rgb]{0.56,0.35,0.01}{\textit{#1}}}
\newcommand{\RegionMarkerTok}[1]{#1}
\newcommand{\SpecialCharTok}[1]{\textcolor[rgb]{0.00,0.00,0.00}{#1}}
\newcommand{\SpecialStringTok}[1]{\textcolor[rgb]{0.31,0.60,0.02}{#1}}
\newcommand{\StringTok}[1]{\textcolor[rgb]{0.31,0.60,0.02}{#1}}
\newcommand{\VariableTok}[1]{\textcolor[rgb]{0.00,0.00,0.00}{#1}}
\newcommand{\VerbatimStringTok}[1]{\textcolor[rgb]{0.31,0.60,0.02}{#1}}
\newcommand{\WarningTok}[1]{\textcolor[rgb]{0.56,0.35,0.01}{\textbf{\textit{#1}}}}
\usepackage{graphicx}
\makeatletter
\def\maxwidth{\ifdim\Gin@nat@width>\linewidth\linewidth\else\Gin@nat@width\fi}
\def\maxheight{\ifdim\Gin@nat@height>\textheight\textheight\else\Gin@nat@height\fi}
\makeatother
% Scale images if necessary, so that they will not overflow the page
% margins by default, and it is still possible to overwrite the defaults
% using explicit options in \includegraphics[width, height, ...]{}
\setkeys{Gin}{width=\maxwidth,height=\maxheight,keepaspectratio}
% Set default figure placement to htbp
\makeatletter
\def\fps@figure{htbp}
\makeatother
\setlength{\emergencystretch}{3em} % prevent overfull lines
\providecommand{\tightlist}{%
  \setlength{\itemsep}{0pt}\setlength{\parskip}{0pt}}
\setcounter{secnumdepth}{-\maxdimen} % remove section numbering
\ifLuaTeX
  \usepackage{selnolig}  % disable illegal ligatures
\fi
\IfFileExists{bookmark.sty}{\usepackage{bookmark}}{\usepackage{hyperref}}
\IfFileExists{xurl.sty}{\usepackage{xurl}}{} % add URL line breaks if available
\urlstyle{same} % disable monospaced font for URLs
\hypersetup{
  pdftitle={PracticaSpam},
  pdfauthor={DavidVelasco},
  hidelinks,
  pdfcreator={LaTeX via pandoc}}

\title{PracticaSpam}
\author{DavidVelasco}
\date{2023-01-25}

\begin{document}
\maketitle

En la detección de Spam se utilizan con frecuencia técnicas de machine
learning para mejorar los índices de detección de correos no deseados.
En el dataset adjunto, se han seleccionado para cada mensaje una serie
de términos clave que suelen aparecer con frecuencia en los mensajes
spam. Posteriormente, se ha realizado una codificación vectorial de los
correos electrónicos considerando esos términos clave. Para cada correo
disponemos de la clasificación por parte de los expertos humanos. Se
pide realizar las siguientes tareas:

\begin{itemize}
\item
  Apartado 1) Sustituir un 2\% de valores de la matriz de datos por NAs,
  de manera aleatoria. Imputar dichos valores faltantes y justificar la
  elección del método utilizado.(1 punto)
\item
  Apartado 2) Realizar un análisis exploratorio de la matriz de datos.
  Comentar los resultados y utilizar visualizaciones cuando sea
  necesario. (2 puntos)
\item
  Apartado 3) Eliminar aquellas palabras que tengan una correlación
  elevada con otras. Calcular el número de documentos en que aparece
  cada palabra y eliminar aquellas de menor frecuencia. Dibujar un
  histograma de dichas frecuencias calculadas. Nota: Con la codificación
  ``bag of words'' utilizada, el número de documentos en que aparece una
  palabra se obtiene sumando para cada variable todas las filas. (2
  puntos)
\item
  Apartado 4) Proyectar los datos sobre un subespacio de dimensión menor
  utilizando PCA. ¿ Cuántas componentes principales se deben utilizar
  para poder visualizar la estructura semántica de los documentos ?
  Obtener un plot utilizando dichas compoentes principales y comentar si
  hay estructura de grupo.(2 puntos)
\item
  Apartado 5) Realizar un clustering de los mensajes de spam atendiendo
  a su contenido semántico. Discutir y comparar el resultado para dos
  algoritmos diferentes.(2 puntos)
\item
  Apartado 6) Aplicar los mapas autoorganizativos para visualizar la
  estructura semántica de la colección de documentos utilizada. ¿ Qué
  ventajas tienen los mapas autoorganizativos con respecto a los
  algoritmos de clustering utilizados anteriormente ? (1 punto)
\end{itemize}

\begin{Shaded}
\begin{Highlighting}[]
\CommentTok{\#Comprobación de que está el *.csv en el directorio}

\NormalTok{currentDir }\OtherTok{\textless{}{-}} \FunctionTok{getwd}\NormalTok{()}
\FunctionTok{list.files}\NormalTok{(}\AttributeTok{path=}\StringTok{"../datos"}\NormalTok{)}
\end{Highlighting}
\end{Shaded}

\begin{verbatim}
##  [1] "anemonefish.xls"          "beer2.csv"               
##  [3] "DatasetLimpio.xlsx"       "EXAMPLE_DataToClean.xlsx"
##  [5] "nombre_variables.txt"     "output"                  
##  [7] "religions.csv"            "spam.csv"                
##  [9] "spam.xls"                 "student-mat.csv"         
## [11] "student-por.csv"          "student.zip"
\end{verbatim}

\begin{Shaded}
\begin{Highlighting}[]
\ControlFlowTok{if}\NormalTok{ (}\SpecialCharTok{!}\FunctionTok{file.exists}\NormalTok{(}\StringTok{"../datos"}\NormalTok{)) }
\NormalTok{\{}\FunctionTok{stop}\NormalTok{(}\FunctionTok{paste0}\NormalTok{(}\StringTok{"Se necesita que el directorio datos esté en: "}\NormalTok{,currentDir))\} }


\NormalTok{ComprobarInputs }\OtherTok{\textless{}{-}} \ControlFlowTok{function}\NormalTok{(path, dir,file)}
\NormalTok{\{}\ControlFlowTok{if}\NormalTok{ (}\SpecialCharTok{!}\FunctionTok{file.exists}\NormalTok{(}\FunctionTok{paste0}\NormalTok{(path,}\StringTok{"/"}\NormalTok{,dir)))}
\NormalTok{\{}\FunctionTok{stop}\NormalTok{(}\FunctionTok{paste0}\NormalTok{(}\StringTok{"Se necesita que el directorio "}\NormalTok{, dir, }\StringTok{" esté en: "}\NormalTok{,path))\}}
  \ControlFlowTok{else} \ControlFlowTok{if}\NormalTok{ (}\SpecialCharTok{!}\FunctionTok{file.exists}\NormalTok{(}\FunctionTok{paste0}\NormalTok{(path, }\StringTok{"/"}\NormalTok{,dir,}\StringTok{"/"}\NormalTok{, file)))}
\NormalTok{    \{}\FunctionTok{stop}\NormalTok{(}\FunctionTok{paste0}\NormalTok{(}\StringTok{"Se necesita que "}\NormalTok{, file,}\StringTok{" esté en: "}\NormalTok{, path, }\StringTok{"/"}\NormalTok{, dir))\}\}}

\NormalTok{parentPath }\OtherTok{\textless{}{-}} \FunctionTok{dirname}\NormalTok{(currentDir)}

\FunctionTok{try}\NormalTok{(}\FunctionTok{ComprobarInputs}\NormalTok{(parentPath,}\StringTok{"datos"}\NormalTok{, }\StringTok{"spam.xls"}\NormalTok{), }\ConstantTok{FALSE}\NormalTok{)}

\DocumentationTok{\#\#Unificacion de los dos ficheros en un dataframe}

\CommentTok{\#Cargo el xls y separo mediante un espacio los registros}
\NormalTok{dtDatos}\OtherTok{=}\FunctionTok{read.table}\NormalTok{(}\StringTok{"../datos/spam.xls"}\NormalTok{,}\AttributeTok{sep=}\StringTok{" "}\NormalTok{,}\AttributeTok{header=}\ConstantTok{FALSE}\NormalTok{)}

\CommentTok{\#Cargo los valores del fichero nombre\_variables que serán el nombre de las columnas}
\NormalTok{dtCabeceraFilas}\OtherTok{=}\FunctionTok{read\_fwf}\NormalTok{(}\StringTok{"../datos/nombre\_variables.txt"}\NormalTok{,}\AttributeTok{show\_col\_types =} \ConstantTok{FALSE}\NormalTok{)}
 
\CommentTok{\#Superpongo las datos de las filas por columnas}
\NormalTok{dtCabeceraColumnas }\OtherTok{\textless{}{-}} \FunctionTok{as.data.frame}\NormalTok{(}\FunctionTok{t}\NormalTok{(dtCabeceraFilas))}

\CommentTok{\#Unifico las dos tablas copiando la fila 1 en las cabecera del dataset final}
\FunctionTok{names}\NormalTok{(dtDatos)}\OtherTok{\textless{}{-}}\NormalTok{ dtCabeceraColumnas[}\DecValTok{1}\NormalTok{,]}
\end{Highlighting}
\end{Shaded}

\begin{Shaded}
\begin{Highlighting}[]
  \CommentTok{\#Apartado 1)}

  \CommentTok{\#Nrow= 4601 y ncol=58}
  \CommentTok{\#nValores a sustituir=4601*58*0,02}

\NormalTok{  constante }\OtherTok{=} \FloatTok{0.02}
\NormalTok{  filas}\OtherTok{\textless{}{-}}\FunctionTok{nrow}\NormalTok{(dtDatos)}
\NormalTok{  columnas}\OtherTok{\textless{}{-}}\FunctionTok{ncol}\NormalTok{(dtDatos)}\SpecialCharTok{{-}}\DecValTok{1} \CommentTok{\#Borro 1 para no poner NA en la ultima columna}
\NormalTok{  nvalores }\OtherTok{\textless{}{-}} \FunctionTok{ceiling}\NormalTok{( filas}\SpecialCharTok{*}\NormalTok{ (columnas}\SpecialCharTok{+}\DecValTok{1}\NormalTok{)  }\SpecialCharTok{*}\NormalTok{ constante)}
  \CommentTok{\#Hay 5338 NA en la matriz}
  
  \CommentTok{\#Duplico el dataframe para hacer este apartado}
\NormalTok{  dtDatosNA}\OtherTok{\textless{}{-}}\NormalTok{dtDatos}

  \CommentTok{\#Pasos y justificación}
  \CommentTok{\#{-}{-}{-}{-}{-}{-}{-}{-}{-}{-}{-}{-}{-}{-}{-}{-}{-}{-}{-}{-}{-}}
  
  \CommentTok{\#1. Recorro mediante un while hasta completar el 2\% de los valores del dataframe}
  \CommentTok{\#2. Obtengo un valor aleatorio entre 1 y el numero de filas y entre 1 y el numero de columnas}
  \CommentTok{\#3. Compruebo si en esa celda hay un NA.}
  \CommentTok{\#3.1 Si es un NA reiniciamos la iteración del bucle}
  \CommentTok{\#3.2 Si no hay NA, seteamos el NA en esa celda e iteramos el bucle}
  
  \CommentTok{\#Nota: La funcion sample(x:y,1) obtiene un(1) nº aleatorio entre \textquotesingle{}x\textquotesingle{} e \textquotesingle{}y\textquotesingle{}}
\NormalTok{  i}\OtherTok{=}\DecValTok{1}
  \ControlFlowTok{while}\NormalTok{(i}\SpecialCharTok{\textless{}=}\NormalTok{nvalores)\{}
\NormalTok{    valorfila }\OtherTok{\textless{}{-}}\FunctionTok{sample}\NormalTok{(}\DecValTok{1}\SpecialCharTok{:}\NormalTok{filas,}\DecValTok{1}\NormalTok{)}
\NormalTok{    valorcol}\OtherTok{\textless{}{-}}\FunctionTok{sample}\NormalTok{(}\DecValTok{1}\SpecialCharTok{:}\NormalTok{columnas,}\DecValTok{1}\NormalTok{)}
    
    \ControlFlowTok{if}\NormalTok{(}\FunctionTok{is.na}\NormalTok{(dtDatosNA[valorfila,valorcol]))\{}
\NormalTok{    \}}\ControlFlowTok{else}\NormalTok{\{}
\NormalTok{      dtDatosNA[valorfila,valorcol]}\OtherTok{=}\ConstantTok{NA}
\NormalTok{      i}\OtherTok{\textless{}{-}}\NormalTok{i}\SpecialCharTok{+}\DecValTok{1}
\NormalTok{    \}}
\NormalTok{  \}}
  \CommentTok{\#Compruebo que tengo los 5338 valores en dtDatosNa}
\NormalTok{  valoresNA }\OtherTok{\textless{}{-}}\FunctionTok{sum}\NormalTok{(}\FunctionTok{is.na}\NormalTok{(dtDatosNA))}
  \FunctionTok{cat}\NormalTok{(}\FunctionTok{paste}\NormalTok{(}\StringTok{"Total valores NA en el dataframe dtDatosNA: "}\NormalTok{, valoresNA,}\AttributeTok{sep=}\StringTok{" "}\NormalTok{))}
\end{Highlighting}
\end{Shaded}

\begin{verbatim}
## Total valores NA en el dataframe dtDatosNA:  5338
\end{verbatim}

\begin{Shaded}
\begin{Highlighting}[]
  \CommentTok{\#Comprobación por columnas}
  \FunctionTok{data.frame}\NormalTok{(}\FunctionTok{lapply}\NormalTok{( }\FunctionTok{lapply}\NormalTok{( dtDatosNA,is.na),sum))}
\end{Highlighting}
\end{Shaded}

\begin{verbatim}
##   make address all X3d our over remove internet order mail receive will people
## 1  115     107  93  86  90   93     99       99    94   97      97   82     75
##   report addresses free business email you credit your font X000 money hp hpl
## 1     81        95   93       96   106  83     85  104  101   99    98 92 100
##   george X650 lab labs telnet X857 data X415 X85 technology X1999 parts pm
## 1     86   94  89   89     71   77   89   79 102        111    97    94 85
##   direct cs meeting original project re edu table conference X. X..1 X..2 X..3
## 1    104 85      89       92     102 91  97    93        108 96  102  107   86
##   X..4 X..5 cap_run_length_average cap_run_length_longest cap_run_length_total
## 1  104   89                     74                     90                  106
##   clase
## 1     0
\end{verbatim}

\begin{Shaded}
\begin{Highlighting}[]
  \CommentTok{\#Ultimo paso: Donde hay NA sustituir valores y justificar el metodo}
  \CommentTok{\#El metodo elegido es na\_interpolation de biblioteca imputeTS}
  \CommentTok{\#Es utili cuando los valores NA estan distribuidos de forma aleatoria y se desea}
  \CommentTok{\#mantener la tendencia general de los datos}

  \FunctionTok{library}\NormalTok{(imputeTS)}
\NormalTok{  dtDatosFilled}\OtherTok{\textless{}{-}}\FunctionTok{na\_interpolation}\NormalTok{(dtDatosNA)}

  \CommentTok{\#Compruebo que no tengo ningun valor NA}
\NormalTok{  valoresNAfil }\OtherTok{\textless{}{-}}\FunctionTok{sum}\NormalTok{(}\FunctionTok{is.na}\NormalTok{(dtDatosFilled))}
  \FunctionTok{cat}\NormalTok{(}\FunctionTok{paste}\NormalTok{(}\StringTok{"Total valores NA en el dataframe dtDatosFilled: "}\NormalTok{, valoresNAfil,}\AttributeTok{sep=}\StringTok{" "}\NormalTok{))}
\end{Highlighting}
\end{Shaded}

\begin{verbatim}
## Total valores NA en el dataframe dtDatosFilled:  0
\end{verbatim}

\begin{Shaded}
\begin{Highlighting}[]
  \CommentTok{\#Destacar que la ultima columna es la que confirma si es o no spam el correo}
  \FunctionTok{dim}\NormalTok{(dtDatosFilled)  }\CommentTok{\#Filas(emails) y columnas(variables)}
\end{Highlighting}
\end{Shaded}

\begin{verbatim}
## [1] 4601   58
\end{verbatim}

\begin{Shaded}
\begin{Highlighting}[]
  \CommentTok{\#Visualizacion}
  
  \CommentTok{\#Vamos a comprobar por ejemplo si business/free/you siguen una distribucio normal}
  \FunctionTok{boxplot}\NormalTok{(dtDatosFilled}\SpecialCharTok{$}\NormalTok{business)}
\end{Highlighting}
\end{Shaded}

\includegraphics{PracticaSpam_files/figure-latex/unnamed-chunk-2-1.pdf}

\begin{Shaded}
\begin{Highlighting}[]
  \FunctionTok{qqnorm}\NormalTok{(dtDatosFilled}\SpecialCharTok{$}\NormalTok{business)}
\end{Highlighting}
\end{Shaded}

\includegraphics{PracticaSpam_files/figure-latex/unnamed-chunk-2-2.pdf}

\begin{Shaded}
\begin{Highlighting}[]
  \FunctionTok{hist}\NormalTok{(dtDatosFilled}\SpecialCharTok{$}\NormalTok{business)}
\end{Highlighting}
\end{Shaded}

\includegraphics{PracticaSpam_files/figure-latex/unnamed-chunk-2-3.pdf}

\begin{Shaded}
\begin{Highlighting}[]
  \FunctionTok{shapiro.test}\NormalTok{(dtDatosFilled}\SpecialCharTok{$}\NormalTok{business)}
\end{Highlighting}
\end{Shaded}

\begin{verbatim}
## 
##  Shapiro-Wilk normality test
## 
## data:  dtDatosFilled$business
## W = 0.3621, p-value < 2.2e-16
\end{verbatim}

\begin{Shaded}
\begin{Highlighting}[]
  \FunctionTok{boxplot}\NormalTok{(dtDatosFilled}\SpecialCharTok{$}\NormalTok{free)}
\end{Highlighting}
\end{Shaded}

\includegraphics{PracticaSpam_files/figure-latex/unnamed-chunk-2-4.pdf}

\begin{Shaded}
\begin{Highlighting}[]
  \FunctionTok{qqnorm}\NormalTok{(dtDatosFilled}\SpecialCharTok{$}\NormalTok{free)}
\end{Highlighting}
\end{Shaded}

\includegraphics{PracticaSpam_files/figure-latex/unnamed-chunk-2-5.pdf}

\begin{Shaded}
\begin{Highlighting}[]
  \FunctionTok{hist}\NormalTok{(dtDatosFilled}\SpecialCharTok{$}\NormalTok{free)}
\end{Highlighting}
\end{Shaded}

\includegraphics{PracticaSpam_files/figure-latex/unnamed-chunk-2-6.pdf}

\begin{Shaded}
\begin{Highlighting}[]
  \FunctionTok{shapiro.test}\NormalTok{(dtDatosFilled}\SpecialCharTok{$}\NormalTok{free)}
\end{Highlighting}
\end{Shaded}

\begin{verbatim}
## 
##  Shapiro-Wilk normality test
## 
## data:  dtDatosFilled$free
## W = 0.3087, p-value < 2.2e-16
\end{verbatim}

\begin{Shaded}
\begin{Highlighting}[]
  \FunctionTok{boxplot}\NormalTok{(dtDatosFilled}\SpecialCharTok{$}\NormalTok{you)}
\end{Highlighting}
\end{Shaded}

\includegraphics{PracticaSpam_files/figure-latex/unnamed-chunk-2-7.pdf}

\begin{Shaded}
\begin{Highlighting}[]
  \FunctionTok{qqnorm}\NormalTok{(dtDatosFilled}\SpecialCharTok{$}\NormalTok{you)}
\end{Highlighting}
\end{Shaded}

\includegraphics{PracticaSpam_files/figure-latex/unnamed-chunk-2-8.pdf}

\begin{Shaded}
\begin{Highlighting}[]
  \FunctionTok{hist}\NormalTok{(dtDatosFilled}\SpecialCharTok{$}\NormalTok{you)}
\end{Highlighting}
\end{Shaded}

\includegraphics{PracticaSpam_files/figure-latex/unnamed-chunk-2-9.pdf}

\begin{Shaded}
\begin{Highlighting}[]
  \FunctionTok{shapiro.test}\NormalTok{(dtDatosFilled}\SpecialCharTok{$}\NormalTok{you)}
\end{Highlighting}
\end{Shaded}

\begin{verbatim}
## 
##  Shapiro-Wilk normality test
## 
## data:  dtDatosFilled$you
## W = 0.84725, p-value < 2.2e-16
\end{verbatim}

\begin{Shaded}
\begin{Highlighting}[]
  \CommentTok{\#Ninguna de ellas sigue una distribucion normal. La más cerca de hacerlo es la variable \textquotesingle{}you\textquotesingle{}}
  
  \CommentTok{\#Correos spam o no spam}
  \FunctionTok{hist}\NormalTok{(dtDatosFilled}\SpecialCharTok{$}\NormalTok{clase,}\AttributeTok{main=}\StringTok{"Cantidad de correos spam"}\NormalTok{,}
       \AttributeTok{col=} \FunctionTok{c}\NormalTok{(}\StringTok{"red"}\NormalTok{,}\StringTok{"blue"}\NormalTok{),}
       \AttributeTok{ylab=}\StringTok{"Suma total"}\NormalTok{,}
       \AttributeTok{xlab=}\StringTok{""}\NormalTok{)}
\end{Highlighting}
\end{Shaded}

\includegraphics{PracticaSpam_files/figure-latex/unnamed-chunk-2-10.pdf}

\begin{Shaded}
\begin{Highlighting}[]
\NormalTok{  p}\OtherTok{\textless{}{-}}\FunctionTok{table}\NormalTok{(dtDatosFilled}\SpecialCharTok{$}\NormalTok{clase)}
  \FunctionTok{prop.table}\NormalTok{(p) }\CommentTok{\#Porcentaje de correos si o no spam}
\end{Highlighting}
\end{Shaded}

\begin{verbatim}
## 
##         0         1 
## 0.6059552 0.3940448
\end{verbatim}

\begin{Shaded}
\begin{Highlighting}[]
  \CommentTok{\#Compruebo correlacion entre variables aleatoriamente mediante graficos}
  \FunctionTok{plot}\NormalTok{(dtDatosFilled[,}\FunctionTok{c}\NormalTok{(}\DecValTok{20}\NormalTok{,}\DecValTok{25}\NormalTok{,}\DecValTok{50}\NormalTok{,}\DecValTok{55}\NormalTok{,}\DecValTok{56}\NormalTok{,}\DecValTok{57}\NormalTok{,}\DecValTok{58}\NormalTok{)]) }\CommentTok{\#No se observa ninguna correlacion entre las variables seleccionadas}
\end{Highlighting}
\end{Shaded}

\includegraphics{PracticaSpam_files/figure-latex/unnamed-chunk-4-1.pdf}

\begin{Shaded}
\begin{Highlighting}[]
  \CommentTok{\#Resumen por variables}
  \FunctionTok{require}\NormalTok{(psych)}
\NormalTok{  psych}\SpecialCharTok{::}\FunctionTok{describe}\NormalTok{(dtDatosFilled}\SpecialCharTok{$}\NormalTok{cap\_run\_length\_total) }
\end{Highlighting}
\end{Shaded}

\begin{verbatim}
##    vars    n   mean     sd median trimmed    mad min   max range skew kurtosis
## X1    1 4601 279.48 560.33     95  156.53 112.68   1 10062 10061 6.39    71.76
##      se
## X1 8.26
\end{verbatim}

\begin{Shaded}
\begin{Highlighting}[]
\NormalTok{  psych}\SpecialCharTok{::}\FunctionTok{describe}\NormalTok{(dtDatosFilled}\SpecialCharTok{$}\NormalTok{free) }
\end{Highlighting}
\end{Shaded}

\begin{verbatim}
##    vars    n mean   sd median trimmed mad min max range  skew kurtosis   se
## X1    1 4601 0.25 0.82      0    0.08   0   0  20    20 10.85   199.14 0.01
\end{verbatim}

\begin{Shaded}
\begin{Highlighting}[]
\NormalTok{  psych}\SpecialCharTok{::}\FunctionTok{describe}\NormalTok{(dtDatosFilled}\SpecialCharTok{$}\NormalTok{money) }
\end{Highlighting}
\end{Shaded}

\begin{verbatim}
##    vars    n mean   sd median trimmed mad min  max range  skew kurtosis   se
## X1    1 4601 0.09 0.44      0    0.01   0   0 12.5  12.5 14.73   303.28 0.01
\end{verbatim}

\begin{Shaded}
\begin{Highlighting}[]
  \CommentTok{\#Apartado 3)}
  \CommentTok{\#La correlacion entre dos variables va desde {-}1 a 1(correlacion negativa o positiva perfecta)}
  \CommentTok{\#Cuanto más cerca este de 1 o {-}1 la correlación es más alta.}
  \CommentTok{\#Atendiendo al apartado anterior donde se representa la correlacion en el cruce de variables}
  \CommentTok{\#Considerando a partir de 0.90(neg. o pos.) una correlacion alta:}
  \CommentTok{\#Se observa correlacion alta para las variables 857 y 415}

\NormalTok{  M}\OtherTok{\textless{}{-}}\FunctionTok{cor}\NormalTok{(}\FunctionTok{Filter}\NormalTok{(is.numeric, dtDatosFilled))}
  
  \FunctionTok{library}\NormalTok{(corrplot)}
\NormalTok{  cor\_matrix }\OtherTok{\textless{}{-}} \FunctionTok{cor}\NormalTok{(dtDatosFilled)}
  
  \CommentTok{\#Considero a partir de 85\% una correlacion alta}
\NormalTok{  listaPalabrasCorr }\OtherTok{\textless{}{-}} \FunctionTok{apply}\NormalTok{(cor\_matrix, }\DecValTok{1}\NormalTok{, }\ControlFlowTok{function}\NormalTok{(x) }\FunctionTok{which}\NormalTok{(x }\SpecialCharTok{\textgreater{}=} \FloatTok{0.85} \SpecialCharTok{\&}\NormalTok{ x }\SpecialCharTok{\textless{}} \DecValTok{1}\NormalTok{))}
\NormalTok{  columnasEliminar }\OtherTok{\textless{}{-}} \FunctionTok{unlist}\NormalTok{(listaPalabrasCorr)}
  \FunctionTok{cat}\NormalTok{(}\FunctionTok{paste}\NormalTok{(}\StringTok{"Las columna a borrar son:"}\NormalTok{))}
\end{Highlighting}
\end{Shaded}

\begin{verbatim}
## Las columna a borrar son:
\end{verbatim}

\begin{Shaded}
\begin{Highlighting}[]
  \FunctionTok{cat}\NormalTok{(}\FunctionTok{paste}\NormalTok{(columnasEliminar,}\AttributeTok{sep=}\StringTok{" "}\NormalTok{))}
\end{Highlighting}
\end{Shaded}

\begin{verbatim}
## 34 32
\end{verbatim}

\begin{Shaded}
\begin{Highlighting}[]
  \FunctionTok{cor}\NormalTok{(dtDatosFilled}\SpecialCharTok{$}\StringTok{\textasciigrave{}}\AttributeTok{857}\StringTok{\textasciigrave{}}\NormalTok{, dtDatosFilled}\SpecialCharTok{$}\StringTok{\textasciigrave{}}\AttributeTok{415}\StringTok{\textasciigrave{}}\NormalTok{)}
\end{Highlighting}
\end{Shaded}

\begin{verbatim}
## [1] 0.9815477
\end{verbatim}

\begin{Shaded}
\begin{Highlighting}[]
  \CommentTok{\#Se borran las columnas 34 y 32 que son las que tienen una correlacion muy alta entre ellas}
\NormalTok{  dtDatosFinal}\OtherTok{\textless{}{-}}\NormalTok{dtDatosFilled}
\NormalTok{  dtDatosFinal}\SpecialCharTok{$}\StringTok{\textasciigrave{}}\AttributeTok{857}\StringTok{\textasciigrave{}}\OtherTok{\textless{}{-}} \ConstantTok{NULL}
\NormalTok{  dtDatosFinal}\SpecialCharTok{$}\StringTok{\textasciigrave{}}\AttributeTok{415}\StringTok{\textasciigrave{}}\OtherTok{\textless{}{-}} \ConstantTok{NULL}
  
\NormalTok{  M2}\OtherTok{\textless{}{-}}\FunctionTok{cor}\NormalTok{(}\FunctionTok{Filter}\NormalTok{(is.numeric, dtDatosFinal)) }\CommentTok{\#La siguiente correlacion mas alta está alrededor del 0.65}
  
  \CommentTok{\#Calculo del nº de documentos en los que aparece una palabra = Suma de cada variable en todas las filas}
\NormalTok{  dtDocumentosPalabra}\OtherTok{\textless{}{-}}\FunctionTok{colSums}\NormalTok{(dtDatosFinal)}
  \FunctionTok{class}\NormalTok{(dtDocumentosPalabra)}
\end{Highlighting}
\end{Shaded}

\begin{verbatim}
## [1] "numeric"
\end{verbatim}

\begin{Shaded}
\begin{Highlighting}[]
\NormalTok{  dfDocumentosPalabra}\OtherTok{\textless{}{-}}\FunctionTok{data.frame}\NormalTok{(dtDocumentosPalabra)}
  
  \CommentTok{\#Ploteo el histograma para ver que variables puedo suprimir}
  \FunctionTok{hist}\NormalTok{(dfDocumentosPalabra}\SpecialCharTok{$}\NormalTok{dtDocumentosPalabra)}
\end{Highlighting}
\end{Shaded}

\includegraphics{PracticaSpam_files/figure-latex/unnamed-chunk-5-1.pdf}

\begin{Shaded}
\begin{Highlighting}[]
  \CommentTok{\#Ordeno de menor a mayor y decido que borro las columnas cuyo valor es inferior a 200}
  \FunctionTok{library}\NormalTok{(dplyr)}
\NormalTok{  dfDocumentosPalabra }\SpecialCharTok{\%\textgreater{}\%} \FunctionTok{arrange}\NormalTok{(dtDocumentosPalabra)}
\end{Highlighting}
\end{Shaded}

\begin{verbatim}
##                        dtDocumentosPalabra
## table                              24.5050
## parts                              63.8850
## [                                  78.6525
## conference                        134.1850
## ;                                 175.1630
## #                                 199.5840
## cs                                200.1300
## original                          214.7200
## addresses                         228.8100
## report                            271.1000
## receive                           278.2750
## telnet                            287.5300
## direct                            302.1950
## 3d                                319.6050
## project                           347.9950
## $                                 350.7780
## pm                                355.0400
## credit                            397.1050
## order                             410.1300
## money                             430.2950
## people                            430.7900
## over                              444.1550
## data                              446.8600
## technology                        449.3600
## lab                               466.7150
## labs                              471.1250
## 000                               474.9300
## 85                                481.9350
## make                              484.3750
## internet                          488.7500
## remove                            523.4650
## font                              565.1500
## 650                               567.0200
## meeting                           595.9500
## 1999                              626.4300
## (                                 639.7745
## business                          648.8400
## edu                               837.6000
## email                             845.0600
## address                           986.6100
## mail                             1100.2600
## free                             1141.2700
## hpl                              1216.7900
## !                                1234.3125
## all                              1297.1300
## re                               1396.2750
## our                              1439.3600
## clase                            1813.0000
## will                             2500.9700
## hp                               2563.7450
## george                           3538.9100
## your                             3733.0650
## you                              7637.1800
## cap_run_length_average          23921.9450
## cap_run_length_longest         239449.5000
## cap_run_length_total          1285894.0000
\end{verbatim}

\begin{Shaded}
\begin{Highlighting}[]
  \CommentTok{\#Borro las columnas con menos frecuencia, son 5: table {-}conference {-} ; {-} parts {-} [}
  
\NormalTok{  dtDatosFinal}\SpecialCharTok{$}\NormalTok{table}\OtherTok{\textless{}{-}} \ConstantTok{NULL}
\NormalTok{  dtDatosFinal}\SpecialCharTok{$}\NormalTok{conference}\OtherTok{\textless{}{-}} \ConstantTok{NULL}
\NormalTok{  dtDatosFinal}\SpecialCharTok{$}\StringTok{\textasciigrave{}}\AttributeTok{;}\StringTok{\textasciigrave{}}\OtherTok{\textless{}{-}} \ConstantTok{NULL}
\NormalTok{  dtDatosFinal}\SpecialCharTok{$}\NormalTok{parts}\OtherTok{\textless{}{-}} \ConstantTok{NULL}
\NormalTok{  dtDatosFinal}\SpecialCharTok{$}\StringTok{\textasciigrave{}}\AttributeTok{[}\StringTok{\textasciigrave{}}\OtherTok{\textless{}{-}} \ConstantTok{NULL}
\NormalTok{  dtDatosFinal}\SpecialCharTok{$}\NormalTok{cs}\OtherTok{\textless{}{-}} \ConstantTok{NULL}
  
  \CommentTok{\#Además, considero que es mejor eliminar las ultimas 4 columnas para observar mejor el histograma}
  \CommentTok{\#de la frecuencia de las palabras y que haya ruido}
  \CommentTok{\#Lo guardo en otro dataframe porque voy a trabajar con dtDatosFinal en los apartados siguientes.}
\NormalTok{  dtDatosFinal2 }\OtherTok{\textless{}{-}}\NormalTok{ dtDatosFinal}
  
\NormalTok{  dtDatosFinal2}\SpecialCharTok{$}\NormalTok{cap\_run\_length\_total}\OtherTok{\textless{}{-}} \ConstantTok{NULL}
\NormalTok{  dtDatosFinal2}\SpecialCharTok{$}\NormalTok{cap\_run\_length\_longest }\OtherTok{\textless{}{-}} \ConstantTok{NULL}
\NormalTok{  dtDatosFinal2}\SpecialCharTok{$}\NormalTok{cap\_run\_length\_average}\OtherTok{\textless{}{-}} \ConstantTok{NULL}
\NormalTok{  dtDatosFinal2}\SpecialCharTok{$}\NormalTok{clase}\OtherTok{\textless{}{-}} \ConstantTok{NULL}
  
  
\NormalTok{  dtDocumentosPalabra}\OtherTok{\textless{}{-}}\FunctionTok{colSums}\NormalTok{(dtDatosFinal2)}
\NormalTok{  dfDocumentosPalabra}\OtherTok{\textless{}{-}}\FunctionTok{data.frame}\NormalTok{(dtDocumentosPalabra)}
  
  \CommentTok{\#Ploteamos}
  \FunctionTok{hist}\NormalTok{(dfDocumentosPalabra}\SpecialCharTok{$}\NormalTok{dtDocumentosPalabra)}
\end{Highlighting}
\end{Shaded}

\includegraphics{PracticaSpam_files/figure-latex/unnamed-chunk-5-2.pdf}

\begin{Shaded}
\begin{Highlighting}[]
  \CommentTok{\#Ploteamos poniendo limites en el eje de las x}
  \FunctionTok{hist}\NormalTok{(dfDocumentosPalabra}\SpecialCharTok{$}\NormalTok{dtDocumentosPalabra,}
       \AttributeTok{breaks =}\FunctionTok{seq}\NormalTok{(}\DecValTok{5}\NormalTok{,}\DecValTok{8500}\NormalTok{,}\DecValTok{150}\NormalTok{))}
\end{Highlighting}
\end{Shaded}

\includegraphics{PracticaSpam_files/figure-latex/unnamed-chunk-5-3.pdf}

\begin{Shaded}
\begin{Highlighting}[]
  \CommentTok{\#Ploteamos de nuevo}
  \FunctionTok{hist}\NormalTok{(dfDocumentosPalabra}\SpecialCharTok{$}\NormalTok{dtDocumentosPalabra,}
       \AttributeTok{breaks =}\FunctionTok{seq}\NormalTok{(}\DecValTok{0}\NormalTok{,}\DecValTok{8500}\NormalTok{,}\DecValTok{200}\NormalTok{),}
       \AttributeTok{col=}\StringTok{"\#0099F8"}\NormalTok{,}
       \AttributeTok{border=}\StringTok{"\#000000"}\NormalTok{,}
       \AttributeTok{ylab=}\StringTok{"Frecuencia"}\NormalTok{,}
       \AttributeTok{xlab=}\StringTok{"Suma de palabras"}\NormalTok{,}
       \AttributeTok{main=}\StringTok{\textquotesingle{}Conteo total\textquotesingle{}}\NormalTok{)}
\end{Highlighting}
\end{Shaded}

\includegraphics{PracticaSpam_files/figure-latex/unnamed-chunk-5-4.pdf}

\begin{Shaded}
\begin{Highlighting}[]
  \CommentTok{\#Observamos que hay una palabra(you) que aparece con diferencia más veces que el resto.}
  \CommentTok{\#La mayoria de palabra están entre las 0 y las 170 apariciones}
\end{Highlighting}
\end{Shaded}

\begin{Shaded}
\begin{Highlighting}[]
  \CommentTok{\#Apartado 4)}
  
  \CommentTok{\#Lo primero antes de realizar un analisis de componentes es normalizar la matriz.}
\NormalTok{  dtDatosFinalNorm }\OtherTok{\textless{}{-}} \FunctionTok{scale}\NormalTok{(dtDatosFinal)}


\NormalTok{  dtDatosFinal\_PCA }\OtherTok{\textless{}{-}} \FunctionTok{prcomp}\NormalTok{(dtDatosFinalNorm[,}\DecValTok{1}\SpecialCharTok{:}\DecValTok{50}\NormalTok{], }\AttributeTok{center =} \ConstantTok{TRUE}\NormalTok{, }\AttributeTok{scale. =} \ConstantTok{TRUE}\NormalTok{)}
  
  \FunctionTok{barplot}\NormalTok{(dtDatosFinal\_PCA}\SpecialCharTok{$}\NormalTok{sdev}\SpecialCharTok{\^{}}\DecValTok{2} \SpecialCharTok{/} \FunctionTok{sum}\NormalTok{(dtDatosFinal\_PCA}\SpecialCharTok{$}\NormalTok{sdev}\SpecialCharTok{\^{}}\DecValTok{2}\NormalTok{), }
        \AttributeTok{names.arg =} \DecValTok{1}\SpecialCharTok{:}\FunctionTok{length}\NormalTok{(dtDatosFinal\_PCA}\SpecialCharTok{$}\NormalTok{sdev), }
        \AttributeTok{xlab =} \StringTok{"Componente principal"}\NormalTok{, }
        \AttributeTok{ylab =} \StringTok{"Porcentaje de varianza explicado"}\NormalTok{, }
        \AttributeTok{main =} \StringTok{"Porcentaje de varianza explicado por componente principal"}\NormalTok{)}
\end{Highlighting}
\end{Shaded}

\includegraphics{PracticaSpam_files/figure-latex/unnamed-chunk-6-1.pdf}

\begin{Shaded}
\begin{Highlighting}[]
  \CommentTok{\#Para ver la legenda de los graficos tienen que ejecutarse las dos sentencias a la vez}
  \FunctionTok{plot}\NormalTok{(dtDatosFinal\_PCA}\SpecialCharTok{$}\NormalTok{x[,}\DecValTok{1}\SpecialCharTok{:}\DecValTok{2}\NormalTok{], }\AttributeTok{col =} \FunctionTok{ifelse}\NormalTok{(dtDatosFinal[,}\DecValTok{50}\NormalTok{] }\SpecialCharTok{==} \DecValTok{1}\NormalTok{, }\StringTok{"red"}\NormalTok{, }\StringTok{"blue"}\NormalTok{), }\AttributeTok{pch =} \DecValTok{16}\NormalTok{, }\AttributeTok{cex =} \DecValTok{1}\NormalTok{, }\AttributeTok{main =} \StringTok{"PCA   con dos componentes principales"}\NormalTok{)}
  \FunctionTok{legend}\NormalTok{(}\StringTok{"topleft"}\NormalTok{, }\AttributeTok{legend =} \FunctionTok{c}\NormalTok{(}\StringTok{"No es spam"}\NormalTok{, }\StringTok{"Es spam"}\NormalTok{), }\AttributeTok{col =} \FunctionTok{c}\NormalTok{(}\StringTok{"blue"}\NormalTok{, }\StringTok{"red"}\NormalTok{), }\AttributeTok{pch =} \DecValTok{16}\NormalTok{)}
\end{Highlighting}
\end{Shaded}

\includegraphics{PracticaSpam_files/figure-latex/unnamed-chunk-6-2.pdf}

\begin{Shaded}
\begin{Highlighting}[]
  \FunctionTok{library}\NormalTok{(scatterplot3d)}
  \FunctionTok{scatterplot3d}\NormalTok{(}\AttributeTok{x =}\NormalTok{ dtDatosFinal\_PCA}\SpecialCharTok{$}\NormalTok{x[,}\DecValTok{1}\SpecialCharTok{:}\DecValTok{3}\NormalTok{], }\AttributeTok{y =} \ConstantTok{NULL}\NormalTok{, }\AttributeTok{z =} \ConstantTok{NULL}\NormalTok{, }\AttributeTok{color =} \FunctionTok{ifelse}\NormalTok{(dtDatosFinal[,}\DecValTok{50}\NormalTok{] }\SpecialCharTok{==} \DecValTok{1}\NormalTok{, }\StringTok{"red"}\NormalTok{, }\StringTok{"blue"}\NormalTok{), }\AttributeTok{pch =} \DecValTok{16}\NormalTok{, }\AttributeTok{cex.symbols =} \DecValTok{1}\NormalTok{, }\AttributeTok{main =} \StringTok{"PCA con tres componentes principales"}\NormalTok{)}
  \FunctionTok{legend}\NormalTok{(}\StringTok{"topright"}\NormalTok{, }\AttributeTok{legend =} \FunctionTok{c}\NormalTok{(}\StringTok{"No es spam"}\NormalTok{, }\StringTok{"Es spam"}\NormalTok{), }\AttributeTok{col =} \FunctionTok{c}\NormalTok{(}\StringTok{"blue"}\NormalTok{, }\StringTok{"red"}\NormalTok{), }\AttributeTok{pch =} \DecValTok{16}\NormalTok{)}
\end{Highlighting}
\end{Shaded}

\includegraphics{PracticaSpam_files/figure-latex/unnamed-chunk-6-3.pdf}

\begin{Shaded}
\begin{Highlighting}[]
  \CommentTok{\#Observando el barplot se observa que las dos primeras componentes principales explican el 18\% y junto con la        \#tercera explican el 22\%. Bajo mi punto de vista con 2 componentes principales ya se puede visualizar la }
  \CommentTok{\#estructura semántica de los doscumentos. }
  
  \CommentTok{\#Ploteando el grafico con 2 y 3 CP se puede ver que si hay estructura de grupo.}
\end{Highlighting}
\end{Shaded}

\begin{Shaded}
\begin{Highlighting}[]
  \CommentTok{\#Apartado 5)}
  \CommentTok{\#Clustering para dos metodos diferentes}
  
  \FunctionTok{library}\NormalTok{(cluster)}
  \FunctionTok{library}\NormalTok{(factoextra)}
  \FunctionTok{library}\NormalTok{(magrittr)}
\end{Highlighting}
\end{Shaded}

\begin{verbatim}
## 
## Attaching package: 'magrittr'
\end{verbatim}

\begin{verbatim}
## The following object is masked from 'package:purrr':
## 
##     set_names
\end{verbatim}

\begin{verbatim}
## The following object is masked from 'package:tidyr':
## 
##     extract
\end{verbatim}

\begin{Shaded}
\begin{Highlighting}[]
  \CommentTok{\#Me quedo solo con los correos spam}
\NormalTok{  dtDatosFinal\_spam }\OtherTok{\textless{}{-}} \FunctionTok{subset}\NormalTok{(dtDatosFinal, clase }\SpecialCharTok{==} \DecValTok{1}\NormalTok{)}
  
  \CommentTok{\#Borro la ultima columna ya que no es necesaria}
\NormalTok{  dtDatosFinal\_spam}\SpecialCharTok{$}\NormalTok{clase}\OtherTok{\textless{}{-}} \ConstantTok{NULL}
  
  \CommentTok{\#Escalo para que todas las variables tengan mismo peso}
\NormalTok{  dtDatosFinal\_spam\_scaled }\OtherTok{\textless{}{-}} \FunctionTok{scale}\NormalTok{(dtDatosFinal\_spam)}
  
  \CommentTok{\#Uso el método del codo para encontrar el nº óptimo de clusters}
  \ControlFlowTok{if}\NormalTok{(}\SpecialCharTok{!}\FunctionTok{require}\NormalTok{(}\StringTok{\textquotesingle{}factoextra\textquotesingle{}}\NormalTok{)) \{}
    \FunctionTok{install.packages}\NormalTok{(}\StringTok{\textquotesingle{}factoextra\textquotesingle{}}\NormalTok{)}
    \FunctionTok{library}\NormalTok{(}\StringTok{\textquotesingle{}factoextra\textquotesingle{}}\NormalTok{)}
\NormalTok{  \}}
  \FunctionTok{fviz\_nbclust}\NormalTok{(dtDatosFinal\_spam\_scaled, kmeans, }\AttributeTok{method=}\StringTok{"wss"}\NormalTok{)}
\end{Highlighting}
\end{Shaded}

\includegraphics{PracticaSpam_files/figure-latex/unnamed-chunk-7-1.pdf}

\begin{Shaded}
\begin{Highlighting}[]
  \CommentTok{\#Según el método del codo he decidido que lo mejor es hacer 3 grupos.}
  
  \CommentTok{\#Como no hay mucha correspondencia entre todas las variables se podrian}
  \CommentTok{\#llegar a hacer 9 grupos pero sabemos que no lo mejor por la varianza que está sin explicar}
  
  
  
  \FunctionTok{library}\NormalTok{(stats)}
  \FunctionTok{library}\NormalTok{(ggplot2)}
  
  \ControlFlowTok{if}\NormalTok{(}\SpecialCharTok{!}\FunctionTok{require}\NormalTok{(}\StringTok{\textquotesingle{}ggplot2\textquotesingle{}}\NormalTok{)) \{}
    \FunctionTok{install.packages}\NormalTok{(}\StringTok{\textquotesingle{}ggplot2\textquotesingle{}}\NormalTok{)}
    \FunctionTok{library}\NormalTok{(}\StringTok{\textquotesingle{}ggplot2\textquotesingle{}}\NormalTok{)}
\NormalTok{  \}}
  
  \CommentTok{\#1{-}er algoritmo: K{-}means}
  
  \FunctionTok{fviz\_cluster}\NormalTok{(}\FunctionTok{kmeans}\NormalTok{(dtDatosFinal\_spam\_scaled,}\AttributeTok{centers=}\DecValTok{3}\NormalTok{, }\AttributeTok{iter.max =} \DecValTok{1500}\NormalTok{, }\AttributeTok{nstart=}\DecValTok{25}\NormalTok{), }\AttributeTok{data=}\NormalTok{dtDatosFinal\_spam\_scaled)}
\end{Highlighting}
\end{Shaded}

\includegraphics{PracticaSpam_files/figure-latex/unnamed-chunk-7-2.pdf}

\begin{Shaded}
\begin{Highlighting}[]
  \CommentTok{\#Haciendo pruebas con distintos parámetros tomamos la decision de eliminar el outlier}
  \CommentTok{\#porque no ayuda a explicar los grupos y está muy lejos de cualquier grupo}
  
\NormalTok{  dtDatosFinal\_spam\_scaled\_sin\_outlier }\OtherTok{\textless{}{-}}\NormalTok{ dtDatosFinal\_spam\_scaled[}\SpecialCharTok{{-}}\DecValTok{1754}\NormalTok{,]}
  \FunctionTok{fviz\_nbclust}\NormalTok{(dtDatosFinal\_spam\_scaled\_sin\_outlier, kmeans, }\AttributeTok{method=}\StringTok{"wss"}\NormalTok{)}
\end{Highlighting}
\end{Shaded}

\includegraphics{PracticaSpam_files/figure-latex/unnamed-chunk-7-3.pdf}

\begin{Shaded}
\begin{Highlighting}[]
  \CommentTok{\#Se disminuye la cantidad de grupos y se mejora la calidad de los grupos}
  
  \FunctionTok{fviz\_cluster}\NormalTok{(}\FunctionTok{kmeans}\NormalTok{(dtDatosFinal\_spam\_scaled\_sin\_outlier,}\AttributeTok{centers=}\DecValTok{3}\NormalTok{, }\AttributeTok{iter.max =} \DecValTok{1500}\NormalTok{, }\AttributeTok{nstart=}\DecValTok{25}\NormalTok{,), }\AttributeTok{data=}\NormalTok{dtDatosFinal\_spam\_scaled\_sin\_outlier, }\AttributeTok{geom =} \StringTok{"density"}\NormalTok{,}
             \AttributeTok{show.centroids =} \ConstantTok{TRUE}\NormalTok{)}
\end{Highlighting}
\end{Shaded}

\includegraphics{PracticaSpam_files/figure-latex/unnamed-chunk-7-4.pdf}

\begin{Shaded}
\begin{Highlighting}[]
  \CommentTok{\#2º Algortimo K{-}Medioides = PAM}
  
  \ControlFlowTok{if}\NormalTok{(}\SpecialCharTok{!}\FunctionTok{require}\NormalTok{(}\StringTok{\textquotesingle{}cluster\textquotesingle{}}\NormalTok{)) \{}
    \FunctionTok{install.packages}\NormalTok{(}\StringTok{\textquotesingle{}cluster\textquotesingle{}}\NormalTok{)}
    \FunctionTok{library}\NormalTok{(}\StringTok{\textquotesingle{}cluster\textquotesingle{}}\NormalTok{)}
\NormalTok{  \}}
  
  \CommentTok{\#Usando otra forma para comprobar el numero ideal de grupos decido que el tamaño ideal es 3}
\NormalTok{  gap\_stat }\OtherTok{\textless{}{-}} \FunctionTok{clusGap}\NormalTok{(dtDatosFinal\_spam\_scaled\_sin\_outlier,}
                    \AttributeTok{FUN =}\NormalTok{ pam,}
                    \AttributeTok{K.max =} \DecValTok{5}\NormalTok{, }\CommentTok{\#Nº Maximos de clusters}
                    \AttributeTok{B =} \DecValTok{5}\NormalTok{) }\CommentTok{\#Nº de veces que se genera conjunto aleatorio}

  \FunctionTok{fviz\_gap\_stat}\NormalTok{(gap\_stat)}
\end{Highlighting}
\end{Shaded}

\includegraphics{PracticaSpam_files/figure-latex/unnamed-chunk-7-5.pdf}

\begin{Shaded}
\begin{Highlighting}[]
  \CommentTok{\#Guardar en kmed con 3 clusters}
\NormalTok{  kmed }\OtherTok{\textless{}{-}} \FunctionTok{pam}\NormalTok{(dtDatosFinal\_spam\_scaled\_sin\_outlier, }\AttributeTok{k =} \DecValTok{3}\NormalTok{)}
  
  \FunctionTok{fviz\_cluster}\NormalTok{(kmed, }\AttributeTok{data =}\NormalTok{ dtDatosFinal\_spam\_scaled\_sin\_outlier)}
\end{Highlighting}
\end{Shaded}

\includegraphics{PracticaSpam_files/figure-latex/unnamed-chunk-7-6.pdf}

\begin{Shaded}
\begin{Highlighting}[]
  \CommentTok{\#Partiendo de que las observaciones son muy dispares y muchas variables para cada observacion}
  \CommentTok{\#Haciendo el clustering solo para los correos que son spam, de los 3 grupos que se ven en el plot,}
  \CommentTok{\#se puede concluir que hay un grupo que es considerado spam consistentemente y los otros dos grupos tienen}
  \CommentTok{\#tendencia; uno de ellos a 100\% ser considerado correo malicioso y otro de ellos que es considerado}
  \CommentTok{\#spam pero tiene ciertos componentes que no lo deja claramente agrupado en esta categoría}
\end{Highlighting}
\end{Shaded}

\begin{Shaded}
\begin{Highlighting}[]
  \CommentTok{\#Apartado 6)}

  \ControlFlowTok{if}\NormalTok{(}\SpecialCharTok{!}\FunctionTok{require}\NormalTok{(}\StringTok{\textquotesingle{}kohonen\textquotesingle{}}\NormalTok{)) \{}
    \FunctionTok{install.packages}\NormalTok{(}\StringTok{\textquotesingle{}kohonen\textquotesingle{}}\NormalTok{)}
    \FunctionTok{library}\NormalTok{(}\StringTok{\textquotesingle{}kohonen\textquotesingle{}}\NormalTok{)}
\NormalTok{  \}}
\end{Highlighting}
\end{Shaded}

\begin{verbatim}
## Loading required package: kohonen
\end{verbatim}

\begin{verbatim}
## 
## Attaching package: 'kohonen'
\end{verbatim}

\begin{verbatim}
## The following object is masked from 'package:purrr':
## 
##     map
\end{verbatim}

\begin{Shaded}
\begin{Highlighting}[]
  \CommentTok{\# Cargamos la matriz con los datos con lo que realicé el apartado anterior.(Matriz sin el outlier)}
  \FunctionTok{data}\NormalTok{(dtDatosFinal\_spam\_scaled\_sin\_outlier)}
\end{Highlighting}
\end{Shaded}

\begin{verbatim}
## Warning in data(dtDatosFinal_spam_scaled_sin_outlier): data set
## 'dtDatosFinal_spam_scaled_sin_outlier' not found
\end{verbatim}

\begin{Shaded}
\begin{Highlighting}[]
  \CommentTok{\# Creamos el mapa autoorganizativo}
\NormalTok{  som\_map }\OtherTok{\textless{}{-}} \FunctionTok{som}\NormalTok{(dtDatosFinal\_spam\_scaled\_sin\_outlier[, }\SpecialCharTok{{-}}\FunctionTok{ncol}\NormalTok{(dtDatosFinal\_spam\_scaled\_sin\_outlier)], }\AttributeTok{grid =} \FunctionTok{somgrid}\NormalTok{(}\DecValTok{8}\NormalTok{, }\DecValTok{8}\NormalTok{, }\StringTok{"rectangular"}\NormalTok{))}

  \CommentTok{\# Visualizamos el mapa}
  \FunctionTok{plot}\NormalTok{(som\_map, }\AttributeTok{type =} \StringTok{"count"}\NormalTok{, }\AttributeTok{main =} \StringTok{"Mapa Autoorganizativo"}\NormalTok{)}
\end{Highlighting}
\end{Shaded}

\includegraphics{PracticaSpam_files/figure-latex/unnamed-chunk-8-1.pdf}

\begin{Shaded}
\begin{Highlighting}[]
  \CommentTok{\#Las ventajas que tienen los mapas autoorganizativos frente a los algoritmos de clustering son:}
  
  \CommentTok{\# 1.Los mapas autoorganizativos preservan la topología de los datos de entrada, lo que significa que los puntos      similares se agrupan juntos en el mapa y los puntos diferentes se separan. Esto puede ser muy útil para              identificar patrones y tendencias complejos en los datos que podrían ser difíciles de detectar con otros             métodos.}
  
  \CommentTok{\# 2.Los mapas autoorganizativos son una buena opción para la exploración de datos y la                               identificación de patrones sin conocimiento previo de la estructura de los datos.}
  
  \CommentTok{\# 3.Fácil de implementar: Los mapas autoorganizativos son relativamente fáciles de implementar y no requieren mucha   configuración previa. Esto los hace accesibles y fáciles de usar para una amplia variedad de usuarios, incluyendo    aquellos sin experiencia previa en aprendizaje automático o análisis de datos.}
\end{Highlighting}
\end{Shaded}


\end{document}
